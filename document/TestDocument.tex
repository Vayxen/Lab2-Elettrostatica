\documentclass{article}

\usepackage[margin=1in]{geometry} % manage page dimensions
\usepackage[utf8]{inputenc} % utf-8 encoding
\usepackage[italian]{babel} % italian default text

\usepackage{import} % import other .tex files
\usepackage{fancyhdr} % custom header and footer
\usepackage[hidelinks]{hyperref} % link references
\usepackage{bookmark} % better references

\usepackage{amsmath, amssymb, amsthm} % math
\usepackage{siunitx} % SI unit
\usepackage{tabularx, booktabs, multirow, longtable} % table enhance
\usepackage{caption, subcaption, graphicx} % figure enhance

\title{Laboratorio di Fisica - Esperienza di Elettrostatica}
\author{Vittorio Strano, Arianna Genuardi, Florinda Tesi, Antonio Riolo, Matteo Romano}
\date{\today}

\begin{document}
    \maketitle

    \tableofcontents

    \newpage
    \normalsize
    \pagenumbering{arabic}
    
    %qui vanno messe le sezioni, nella forma
    
    \section{Introduzione}
    {\fontsize{12}{14}\selectfont
Con questo esperimento si vogliono analizzare le caratteristiche di un condensatore a facce piane e parallele, in particolare come la capacità $C$ dipenda dalla quantità di carica $Q$ e dalla differenza di potenziale $V$ seguendo l'\autoref{eq:C(Q,V)}

\begin{equation} \label{eq:C(Q,V)}
    C = \frac{Q}{V}
\end{equation}

oltre che dalla costante dielettrica del mezzo tra le armature $\varepsilon$ e dalla distanza tra le facce $d$ che, in approssimazione $d \ll \sqrt{A}$, segue l'equazione l'\autoref{eq:capacitacondensatore},

\begin{equation} \label{eq:capacitacondensatore}
    C = \varepsilon \frac{A}{d}
\end{equation}

dove $A$ indica la superficie delle armature.

Per stimare la capacità dell'\emph{ice pail}, approssimata a quella di un condensatore cilindrico, è stata utilizzata l'\autoref{eq:capacitagabbia}, 

\begin{equation} \label{eq:capacitagabbia}
    C = \dfrac{2 \pi \varepsilon L}{\ln{\left(\dfrac{D_2}{D_1}\right)}}
\end{equation}

dove $L =$ altezza, $D_1 =$ diametro interno e $D_2 =$ diametro esterno dell'\emph{ice pail}.
\par}
 
    
    \section{Metodi sperimentali}
    \subsection{Strumentazione}
    {\fontsize{11}{14}\selectfont 
Gli strumenti utilizzati in questa esperienza sono:
\begin{enumerate}
    \item \textbf{Calibro ventesimale} con un errore di $0.01 cm$
    \item \textbf{Condensatore a facce piane e parallele ES-9079} con facce circolari di diametro $(17.8 \pm 0.1) cm$ e superficie $A = (249 \pm 3) cm^2$. La facce sono mobili e poste su un binario dotato di riga graduata con errore $0.5 mm$ e distanza massima $d = 10 cm$
    \item \textbf{Elettrometro ES-9078A} con errore di precisione dell'$1\%$ sul f.s. + 1 digit, dotato delle seguenti scale:(1, 3, 10, 30, 100)V e costituito da una capacità interna del valore tabulato $C_e = 27 pF$
    \item \textbf{Generatore di tensione ES-9077} con d.d.p. fissate a $30 V$ $\pm 5\%$ e $1$, $2$ e $3 kV$ $\pm 10\%$
    \item \textbf{Sfera conduttrice ES-9059}
    \item \textbf{\emph{Proof plane}} di diametro $(3.20 \pm 0.01) cm$
    \item \textbf{\emph{Ice pail}} la cui capacità è stata calcolata seguendo l'\autoref{eq:capacitagabbia} dopo aver misurato con il calibro $L = (15.54 \pm 0.01) cm$, $D_1 = (9.94 \pm 0.01) cm$ e $D_2 = (14.75 \pm 0.17) cm$, che essendo non uniforme, è stata calcolata come media e semi-dispersione del massimo $D_{2 \; max} = 14.92 cm$ e del minimo $D_{2 \; min} = 14.58 cm$
    \item \textbf{Cavi} a bassa capacità
\end{enumerate}

\par} 
    \subsection{Procedimento Capacità Elettrometro}
    \subsection{Procedimento Parte I}
    \subsection{Procedimento Parte II}
    \subsection{Procedimento Parte III}
    \subsection{Procedimento Parte IV}
    
    %{\fontsize{11}{14}\selectfont 
Gli strumenti utilizzati in questa esperienza sono:
\begin{enumerate}
    \item \textbf{Calibro ventesimale} con un errore di $0.01 cm$
    \item \textbf{Condensatore a facce piane e parallele ES-9079} con facce circolari di diametro $(17.8 \pm 0.1) cm$ e superficie $A = (249 \pm 3) cm^2$. La facce sono mobili e poste su un binario dotato di riga graduata con errore $0.5 mm$ e distanza massima $d = 10 cm$
    \item \textbf{Elettrometro ES-9078A} con errore di precisione dell'$1\%$ sul f.s. + 1 digit, dotato delle seguenti scale:(1, 3, 10, 30, 100)V e costituito da una capacità interna del valore tabulato $C_e = 27 pF$
    \item \textbf{Generatore di tensione ES-9077} con d.d.p. fissate a $30 V$ $\pm 5\%$ e $1$, $2$ e $3 kV$ $\pm 10\%$
    \item \textbf{Sfera conduttrice ES-9059}
    \item \textbf{\emph{Proof plane}} di diametro $(3.20 \pm 0.01) cm$
    \item \textbf{\emph{Ice pail}} la cui capacità è stata calcolata seguendo l'\autoref{eq:capacitagabbia} dopo aver misurato con il calibro $L = (15.54 \pm 0.01) cm$, $D_1 = (9.94 \pm 0.01) cm$ e $D_2 = (14.75 \pm 0.17) cm$, che essendo non uniforme, è stata calcolata come media e semi-dispersione del massimo $D_{2 \; max} = 14.92 cm$ e del minimo $D_{2 \; min} = 14.58 cm$
    \item \textbf{Cavi} a bassa capacità
\end{enumerate}

\par}
    \section{Risultati}
    \subsection{Risultati Capacità Elettrometro}
    \subsection{Risultati Parte I}
    \subsection{Risultati Parte II}
    \subsection{Risultati Parte III}
    \subsection{Risultati Parte IV}
    
    \section{Conclusioni}
    
    
    
    \appendix
    \section*{Appendices}
    \addcontentsline{toc}{section}{Appendices}
    \renewcommand{\thesubsection}{\Alph{subsection}}
    
    \subsection{Appendix Subsection}
    {\fontsize{11}{14}\selectfont 

Per stimare la capacità del sistema formato dall'elettrometro più i cavi di connessione, è stato collegato il generatore impostato a $(30.0 \pm 1.5) V$ (per l'errore è stato utilizzato un errore massimo come descritto nel manuale $30V \pm 5\%$) al condensatore con distanza tra le armature $d = (0.3 \pm 0.1) cm$. Una volta caricato il condensatore, il generatore è stato scollegato con l'ausilio di guanti, in modo da evitare dispersione di cariche. 
\par
Per misurare la tensione ai capi del condensatore sono stati attaccati i morsetti dell'elettrometro. La tensione misurata, a parità di carica, è $V_s = (12.0 \pm 1.3) V$.
\par
Infine per il calcolo della capacità e del suo errore massimo è stata usata l'\autoref{eq:capacitasistema} e la relativa propagazione degli errori massimi.

\begin{equation} \label{eq:capacitasistema}
    C_{sistema} = C \cdot \dfrac{V - V_s}{V_s} = (113 \pm 30) pF
\end{equation}


\par
Quindi è stato possibile ricavare la capacità dei cavi ed il corrispondente errore massimo come:

\begin{equation}
    C_{cavi} = C_{sistema} - C_e = (86 \pm 30) pF
\end{equation}
\par}
\end{document}