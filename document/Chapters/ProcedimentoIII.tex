{\fontsize{12}{14}\selectfont 

È stata collegata l'uscita di $3kV$ del generatore al condensatore (ground sulla piastra fissa e lead sulla piastra mobile) in modo da avere una tensione costante ai capi del condensatore. L'elettrometro è stato connesso alla gabbia di Faraday (ground sulla superficie esterna e lead su quella interna). Si è quindi proceduto toccando la superficie interna del condensatore con la \emph{proof plane} a tre distanze $r$ dal centro: $r = 0$, $r = \dfrac{R}{2}$ e $r = R$, per poi spostare velocemente la \emph{proof plane} all'interno dell'\emph{ice pail} poco oltre la metà della sua altezza ma senza metterli a contatto. Durante la lettura è stato notato come la \emph{proof plane} tendesse a perdere carica, per cui la misura di tensione è stata effettuata immediatamente dopo l'inserimento della \emph{proof plane}.
\par
Prima di effettuare la misura successiva si è prestato attenzione a scaricare la \emph{proof plane} ed azzerare l'elettrometro.
\par
Si è ripetuto questo procedimento per diverse distanze $d$ tra le piastre del condensatore.
\par
Il plot della tensione (ai capi dell'\emph{ice pail}) in funzione di $d$ è accompagnato da una simulazione in cui si suppone che la capacità dei cavi, che collegano il condensatore al generatore, sia di $(10 \pm 1)pF$.

La distanza minima tra le piastre è stata scelta in modo da permettere all'operatore di muovere velocemente la \emph{proof plane} dalla parte interna della piastra del condensatore all'interno dell'\emph{ice pail} evitando di toccare una seconda volta le facce del condensatore. 
\par
L'intervallo di misura più conveniente risulta essere tra i $3cm$ ed i $5cm$. Per distanze minori, a causa della mancanza di spazio, è richiesta una maggior precisione per muovere la \emph{proof plane} al di fuori delle piastre; questo allungamento dei tempi causa una perdita di carica. Per distanze maggiori la capacità del condensatore diventa molto minore rispetto a quella dei cavi, con una conseguente variazione di tensione trascurabile.
\par}