{\fontsize{11}{14}\selectfont 

Per stimare la capacità del sistema formato dall'elettrometro più i cavi di connessione, è stato collegato il generatore impostato a $(30.0 \pm 1.5) V$ (per l'errore è stato utilizzato un errore massimo come descritto nel manuale $30V \pm 5\%$) al condensatore con distanza tra le armature $d = (0.3 \pm 0.1) cm$. Una volta caricato il condensatore, il generatore è stato scollegato con l'ausilio di guanti, in modo da evitare dispersione di cariche. 
\par
Per misurare la tensione ai capi del condensatore sono stati attaccati i morsetti dell'elettrometro. La tensione misurata, a parità di carica, è $V_s = (12.0 \pm 1.3) V$.
\par
Infine per il calcolo della capacità e del suo errore massimo è stata usata l'\autoref{eq:capacitasistema} e la relativa propagazione degli errori massimi.

\begin{equation} \label{eq:capacitasistema}
    C_{sistema} = C \cdot \dfrac{V - V_s}{V_s} = (113 \pm 30) pF
\end{equation}


\par
Quindi è stato possibile ricavare la capacità dei cavi ed il corrispondente errore massimo come:

\begin{equation}
    C_{cavi} = C_{sistema} - C_e = (86 \pm 30) pF
\end{equation}
\par}