{\fontsize{12}{14}\selectfont 

La parte IV dell'esperienza consiste nel misurare la costante dielettrica della lastra di legno. Per tenere il dielettrico parallelo alle piastre del condensatore, quest'ultimo è stato inclinato in modo da poter poggiare il dielettrico sulla faccia connessa a massa.

\par
Si è quindi collegato l'elettrometro al condensatore e l'uscita del generatore di $3kV$ alla sfera, posizionata ad almeno $50cm$ dal condensatore in modo da evitare processi di induzione che potessero influenzare la misurazione della tensione.
\\
Con la \emph{proof plane} è stata toccata prima la sfera e poi il condensatore per trasferire la carica, facendo in modo che le superfici fossero tangenti e che il contatto avvenisse sempre nello stesso punto. Dopo aver toccato il condensatore è stata rimossa la carica residua dalla \emph{proof plane} mettendola a contatto con la massa.
\par
Questo procedimento è stato ripetuto per varie distanze tra le piastre, per poi mettere questi dati in \autoref{fig:parteIV}.

\par
%review qua sotto? ho fatto un paio di correzioni
Infine per misurare la costante dielettrica sono state prese due misure, una con il dielettrico ed una senza il dielettrico, con una distanza fissata tra le piastre $d = 1.3cm$. La distanza scelta è la più piccola che permette al dielettrico di non essere a contatto con entrambe le piastre in modo da evitare il passaggio di carica. Lo strato d'aria compreso tra le facce del condensatore è stato trascurato nei calcoli.

\par
Si è poi ottenuta la costante dielettrica relativa insieme al suo errore massimo come:

\begin{equation*}
    \varepsilon_r = \dfrac{C_{sistema} \cdot (V_i-V_f)}{C \cdot V_f} + \dfrac{V_i}{V_f} = (3.9 \pm 1.7)
\end{equation*}

dove $V_i$ corrisponde alla tensione letta dall'elettrometro senza il dielettrico, mentre $V_f$ è la tensione letta con il dielettrico all'interno del condensatore.
\par
La formula è stata ottenuta seguendo il manuale indicato in \autoref{link:pascomanual}.
\par}