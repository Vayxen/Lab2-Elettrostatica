{\fontsize{12}{14}\selectfont 

\begin{figure}[H]
  \centering
  \includegraphics[width=13cm]{Figures/Grafico_Parte2.pdf}
  \caption{Grafico della tensione (in Volt) in funzione del numero di tocchi per due set differenti. L'errore sulla tensione è pari all'$1\%$ del f.s. più $1$ digit. Le simulazioni si avvicinano all'andamento del rispettivo set di dati ma non lo descrivono esattamente. Nel fit lineare si nota come gli ultimi punti si trovano al di sotto della retta mentre i primi al di sopra di essa, evidenziando come la carica trasferita diminuisca ad ogni tocco. Il fit è nella forma $V = \text{pendenza} \cdot \text{tocchi}$, quindi passante per lo 0.}
  \label{fig:GraficoParteII}
\end{figure}

\begin{figure}[H]
  \centering
  \includegraphics[width=11cm]{Figures/Grafico_Parte2_Carica_Trasferita.pdf}
  \caption{Istogramma della tensione (in Volt) ai capi del condensatore dopo un tocco. Ci si aspetta una distribuzione gaussiana dei dati che tramite un fit risulta caratterizzata da $\mu = 3.141$ e $\sigma = 0.246$.}
  \label{fig:GraficoParteIICarica}
\end{figure}

Quindi la carica trasferita con il suo errore statistico (ottenuto propagando in quadratura gli errori) è risultata essere:

\begin{equation*}
    Q = \left(C_{sistema} + \dfrac{\varepsilon \cdot A}{d}\right) \cdot V = (47 \pm 5)\cdot 10^{-11} C
\end{equation*}



\par}