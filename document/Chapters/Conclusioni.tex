{\fontsize{12}{14}\selectfont 

In questo esperimento è stato possibile studiare un sistema elettrostatico costituito da un condensatore con distanza tra le armature regolabile, esaminando sia il range di distanze in cui vale l'approssimazione di facce piane e parallele, sia per distanze maggiori. 
\par
È stato verificato come la carica si distribuisca sulle piastre del condensatore e che tocchi successivi della \emph{proof plane} trasferiscono meno carica dei precedenti, come atteso dalla teoria.
\par
Per la costruzione dell'istogramma relativo alla carica trasferita da un singolo tocco sarebbe stato opportuno prendere più misure in modo da verificare con una maggiore accuratezza che l'andamento seguisse quello di una gaussiana.
\par
Per alcune parti dell'esperimento sarebbe stato preferibile avere più set di misure per poterne verificare la ripetibilità o per analizzare il comportamento del sistema alla variazione dei parametri al contorno.
\par
Un accorgimento (che non si è sempre riusciti a seguire) è quello di adattare il fondoscala dell'elettrometro al valore misurato.
\par}
