{\fontsize{12}{14}\selectfont 

Una volta collegato l'elettrometro al condensatore, è stata fissata la distanza tra le piastre a $6mm$ in modo che la sua capacità fosse comparabile con quella del sistema e tale da minimizzare l'errore relativo sulla distanza $d$.
\\
Il generatore è stato poi connesso alla sfera conduttrice, posizionata ad almeno $50cm$ dal condensatore in modo che non fosse presente una carica aggiuntiva dovuta ad un processo di induzione. 
\par
Utilizzando la \emph{proof plane}, la carica è stata trasferita al condensatore.
\\
Sono stati effettuati due set di misure: nel primo l'uscita del generatore è stata selezionata a 1kV in modo da poter fare più tocchi prima di raggiungere la portata dell'elettrometro, mentre nel secondo set è stata scelta l'uscita del generatore a $3kV$.
\par
In seguito sono state plottate le tensioni misurate in funzione del numero di tocchi, insieme alle simulazioni effettuate in precedenza. Nel valutare l'andamento atteso sono state elaborate due ipotesi: la prima prevede un andamento lineare della carica trasferita, la seconda invece una diminuzione della carica trasferita dai tocchi successivi a causa  dell'incremento della densità di carica sul condensatore.
\par
Per valutare la carica trasferita dalla singola bacchettata, con il generatore impostato ad $1kV$, è stata ripetuta 50 volte la misura di tensione di un tocco, scaricando il condensatore e la \emph{proof plane} dopo ogni misurazione.
\\
È stato fatto un istogramma di questi dati ed è stato confrontato con un andamento di tipo gaussiano.
\par}