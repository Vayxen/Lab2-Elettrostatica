{\fontsize{12}{14}\selectfont
Con questo esperimento si vogliono analizzare le caratteristiche di un condensatore a facce piane e parallele e in particolare come la capacità $C$ dipenda dalla quantità di carica $Q$ e dalla differenza di potenziale $V$ seguendo l'\autoref{eq:C(Q,V)},

\begin{equation} \label{eq:C(Q,V)}
    C = \frac{Q}{V}
\end{equation}

oltre che dalla costante dielettrica del mezzo tra le armature $\varepsilon$ e dalla distanza tra le facce $d$ che, in approssimazione $d \ll \sqrt{A}$, segue l'equazione l'\autoref{eq:capacitacondensatore},

\begin{equation} \label{eq:capacitacondensatore}
    C = \varepsilon \frac{A}{d}
\end{equation}

dove $A$ indica la superficie delle armature.

Inoltre è necessaria la formula per la capacità di un condensatore cilindrico (\autoref{eq:capacitagabbia}), utile per il calcolo della capacità della gabbia di Faraday.

\begin{equation} \label{eq:capacitagabbia}
    C = \dfrac{2 \pi \varepsilon L}{\ln{\left(\dfrac{D_2}{D_1}\right)}}
\end{equation}

Dove $L =$ altezza, $D_1 =$ diametro interno e $D_2 =$ diametro esterno dell'\emph{ice pail}.
\par}
