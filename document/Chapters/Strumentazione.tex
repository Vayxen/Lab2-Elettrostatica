{\fontsize{11}{14}\selectfont 
Gli strumenti utilizzati in questa esperienza sono:
\begin{enumerate}
    \item \textbf{Calibro ventesimale} con un errore di $0.01 cm$
    \item \textbf{Condensatore a facce piane e parallele ES-9079} con facce circolari di diametro $(17.8 \pm 0.1) cm$ e superficie $A = (249 \pm 3) cm^2$. La facce sono mobili e poste su un binario dotato di riga graduata con errore $0.5 mm$ e distanza massima $d = 10 cm$
    \item \textbf{Elettrometro ES-9078A} con errore di precisione dell'$1\%$ sul f.s. + 1 digit, dotato delle seguenti scale:(1, 3, 10, 30, 100)V e costituito da una capacità interna del valore tabulato $C_e = 27 pF$
    \item \textbf{Generatore di tensione ES-9077} con d.d.p. fissate a $30 V$ $\pm 5\%$ e $1$, $2$ e $3 kV$ $\pm 10\%$
    \item \textbf{Sfera conduttrice ES-9059}
    \item \textbf{\emph{Proof plane}} di diametro $(3.20 \pm 0.01) cm$
    \item \textbf{\emph{Ice pail}} la cui capacità è stata calcolata seguendo l'\autoref{eq:capacitagabbia} dopo aver misurato con il calibro $L = (15.54 \pm 0.01) cm$, $D_1 = (9.94 \pm 0.01) cm$ e $D_2 = (14.75 \pm 0.17) cm$, che essendo non uniforme, è stata calcolata come media e semi-dispersione del massimo $D_{2 \; max} = 14.92 cm$ e del minimo $D_{2 \; min} = 14.58 cm$
    \item \textbf{Cavi} a bassa capacità
\end{enumerate}

\par}